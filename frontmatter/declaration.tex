\makeatletter
\newcommand{\unchapter}[1]{%
	\begingroup
	\let\@makeschapterhead\@gobble % make \@makechapterhead do nothing
	\chapter*{#1}
	\endgroup
}
\makeatother

\unchapter{Declaration}
\begin{center}
\includegraphics[width=10cm]{USlogo-top}
\addcontentsline{toc}{chapter}{Declaration}
\makeatletter\@mkboth{}{Declaration}\makeatother
\\
{\rmfamily \bfseries \large Plagiaatverklaring / \textit{Plagiarism Declaration} \par}

\begin{enumerate}
	\item Plagiaat is die oorneem en gebruik van die idees, materiaal en ander intellektuele
	eiendom van ander persone asof dit jou eie werk is.\\
	\textit{Plagiarism is the use of ideas, material and other intellectual property of another’s work
	and to present is as my own.}
	
	\item Ek erken dat die pleeg van plagiaat 'n strafbare oortreding is aangesien dit ‘n vorm van
	diefstal is.\\
	\textit{I agree that plagiarism is a punishable offence because it constitutes theft.}
	
	\item Ek verstaan ook dat direkte vertalings plagiaat is.\\
	\textit{I also understand that direct translations are plagiarism.}
	
	\item Dienooreenkomstig is alle aanhalings en bydraes vanuit enige bron (ingesluit die
	internet) volledig verwys (erken). Ek erken dat die woordelikse aanhaal van teks
	sonder aanhalingstekens (selfs al word die bron volledig erken) plagiaat is.\\
	\textit{Accordingly all quotations and contributions from any source whatsoever (including the
	internet) have been cited fully. I understand that the reproduction of text without
	quotation marks (even when the source is cited) is plagiarism.}
	
	\item Ek verklaar dat die werk in hierdie skryfstuk vervat, behalwe waar anders aangedui, my
	eie oorspronklike werk is en dat ek dit nie vantevore in die geheel of gedeeltelik
	ingehandig het vir bepunting in hierdie module/werkstuk of ‘n ander module/werkstuk
	nie.\\
	\textit{I declare that the work contained in this assignment, except where otherwise stated, is
	my original work and that I have not previously (in its entirety or in part) submitted it for
	grading in this module/assignment or another module/assignment.}
\end{enumerate}

\begin{table}
\begin{tabular} { | p{9.1cm} | p{6cm} | }
	\hline
	& \\
	& \\
	 \textbf{ Studentenommer / \textit{Student number}} & \textbf{ Handtekening / \textit{Signature}} \\
	\hline
	& \\
	& \\
	\textbf{ Voorletters en van / \textit{Initials and surname }} & \textbf{ Datum / \textit{Date}} \\
	\hline
\end{tabular}
\end{table}

\end{center}