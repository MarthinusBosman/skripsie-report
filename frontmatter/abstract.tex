\chapter*{Abstract}
\addcontentsline{toc}{chapter}{Abstract}
\makeatletter\@mkboth{}{Abstract}\makeatother

\subsubsection*{English}

One of the fundamental difficulties in speech recognition is the task of extracting useful features from the highly variable time domain signal due to different speakers, tones, channels and acoustic conditions.
In most state-of-the-art computer vision systems, convolutional neural networks (CNNs) are used to automatically learn how to extract relevant features.
In this study, we aim to evaluate how general these features are.
Specifically, we evaluate the features extracted from a trained vision CNN on speech spectrograms against existing techniques such as filter banks and Mel-frequency Cepstral Coefficients (MFCCs).
The features were evaluated against each other using dynamic time warping to achieve an average precision score per feature.
We achieved an average precision (AP) score of X, improving feature extraction significantly from both existing methods.
Furthermore, we present some insight into the features extracted by the model as well as a new technique for High-throughput computing using Google Cloud Platform.

\selectlanguage{afrikaans}

\subsubsection*{Afrikaans}

Een van die fundamentele probleme in spraakherkenning is om nuttige eienskappe vanuit die hoogs veranderlike klankgolf te onttrek weens verskillende sprekers, toonhoogtes, kanale en akoestiese toestande.
In meeste moderne rekenaarvisie netwerke word konvolusionele neurale netwerke (kNN) gebruik om automaties te leer hoe om relevante eienskappe te onttrek.
In hierdie studie beoog ons om te evalueer hoe algemeen hierdie kenmerke is.
Ons evalueer spesifiek die eienskappe wat deur 'n opgeleide visie kNN op spraakspektrogramme verkry is teen bestaande tegnieke soos filterbanke en Melfrekwensie Cepstral Koëffisiënte.
Die eienskappe was evalueer teen mekaar deur dinamiese tydverduistering te gebruik om 'n algemene akkuraatheid punt te behaal.
Die top kNN eienskappe het 'n algemene akkuraatheidspunt van X behaal, wat 'n verbetering is op beide bestaande metodes.
Verder bied ons ook insig in die eienskappe wat deur die model verkry word sowel as 'n nuwe tegniek vir Hoë-deursettingsberekening deur gebruik te maak van Google Cloud Platform.

\selectlanguage{english}