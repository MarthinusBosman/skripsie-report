\chapter*{Abstract}
\addcontentsline{toc}{chapter}{Abstract}
\makeatletter\@mkboth{}{Abstract}\makeatother

\subsubsection*{English}

A fundamental difficulty in speech recognition is finding useful features from the highly variable time domain signal due to different speakers, tones, channels and acoustic conditions.
In most state-of-the-art computer vision systems, convolutional neural networks (CNNs) are used to automatically learn how to extract relevant features.
We aim to evaluate whether these features can be applied to a completely different domain.
Specifically, we evaluate features extracted directly from a trained vision CNN on speech spectrograms against existing techniques such as filterbanks and mel-frequency cepstral coefficients (MFCCs).
Features are evaluated against each other using dynamic time warping to achieve an average precision (AP) score.
We achieve a top AP score 8\% higher than MFCCs.
We present some insight into the features extracted by the model as well as a new technique for high-throughput computing using Google Cloud Platform.

\selectlanguage{afrikaans}

\subsubsection*{Afrikaans}

'n Fundamentele probleem in spraakherkenning is om nuttige kenmerkvektore vanuit die hoogs veranderlike klankgolf, weens verskillende sprekers, toonhoogtes, kanale en akoestiese toestande, te onttrek.
In meeste moderne rekenaarvisie netwerke word konvolusionele neurale netwerke (KNN) gebruik om automaties te leer hoe om relevante kenmerke te onttrek.
Ons beoog om te evalueer of hierdie kenmerke toegepas kan word op 'n heeltemal ander domein.
Ons evalueer spesifiek die kenmerkvektore wat deur 'n opgeleide visie KNN op spraakspektrogramme verkry is teen bestaande tegnieke soos filterbanke en melfrekwensie kepstraalko\"{e}ffisi\"{e}nte (MKKs).
Die kenmerkvektore is evalueer teen mekaar deur dinamiese tydverduistering te gebruik om 'n algemene akkuraatheid punt te behaal.
Die top KNN eienskap het 'n algemene akkuraatheidspunt 8\% ho\"{e}r as MKKs bereik.
Ons bied insig in die kenmerkvektore wat deur die model verkry word sowel as 'n nuwe tegniek vir ho\"{e}-deursettingsberekening deur gebruik te maak van Google Cloud Platform.

\selectlanguage{english}